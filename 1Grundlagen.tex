\section{Surface effects and secondary ion emission}

In comparison to the known secondary electron emission there also exists \textbf{s}econdary \textbf{i}on \textbf{e}mission (SIE).
Theoretical studies of surface ionization are mostly devoted to the production of positive ions from incident atoms having thermal energies \cite{surf_ioniz}. 
The degree of ionization can be derived by applying thermodynamic arguments.
The ionization coefficient of M is then given by
\begin{align}
    \alpha^+(M^+) &= \frac{n^+}{n} \nonumber\\
                  &= \frac{1-r^+}{1-r} \cdot \frac{w^+}{w} \mathrm{exp}\left(
                    \frac{\bar{\Phi}^+ + e \sqrt{eF} - I(M)}{k_B T} \right) \; ,
    \label{eq:pos_inj}
\end{align}
where $n^+$ and $n$ are the numbers of $M^+$ and $M$ coming from unit surface area per unit time, $w^+/w$ is the statistical weight ratio of $M^+$ to $M$, $r^+$ and $r$ are the internal reflection coefficients at the potential barrier on the emitter surface, $\bar{\Phi}^+$ is the average work function,  $T$ is the absolute temperature at the surface, $F$ is the externally applied field and $I(M)$ is the initial energy of the impinging atoms. 
 
With little adaptations I can transfer equation (\ref{eq:pos_inj}) for treating a surface which emits negative ions.
I assume a negatively biased surface like the cathode in an asymmetric ccrf discharge where the equilibrium
\begin{align}
    X + e^- \; \mathrm{(in\; the\; substrate\; metal)} \leftrightharpoons X^-
\end{align}
is attained at the surface.
Hence, the equation for the negative ionization coefficient follows in analogy to equation (\ref{eq:pos_inj})
\begin{align}
    \alpha^-(X^-) &= \frac{n^-}{n} \nonumber\\
                  &= \frac{1-r^-}{1-r} \cdot \frac{w^-}{w} \mathrm{exp}\left( 
                    \frac{-\bar{\Phi}^- + e \sqrt{eF} + A(X)}{k_B T} \right) \; ,
    \label{eq:neg_inj}
\end{align}
where $A(X)$ is the electron affinity of atom $X$, $\bar{\Phi}^-$ is the average work function effective for producing the negative ion $X^-$ on the metal surface and the other parameter are in analogy to the positive ion injection.
To the best of my knowledge. there are no detailed theoretical and experimental studies of the reflection coefficients $r$ for negative ions. \\ 
Therefore no detailed information about the cross-sections for such processes exists.
In this work I assume that negative ions are produced by a positive ion beam on the surface with an efficiency of $\eta=n_-/n_+$.\\ 
The additional flux of negatively charged particles leads to a reduction of the potential drop as in the case of secondary electron emission.

To get a microscopic knowledge of a ccrf discharge the discharge is simulated.
Since nonthermal low pressure plasmas are not collision dominated the particles are not Maxwell distributed.
The mean free paths are of the same magnitude as the electrode gap.
This means that fluid models fail at this point and kinetic models have to be applied.
I will use the PIC-MCC method which is widely used for plasma simulation.


