%%%%%%%%%%%%%%%%%%%%%%% file template.tex %%%%%%%%%%%%%%%%%%%%%%%%%
%
% This is a template file for The European Physical Journal
%
% Copy it to a new file with a new name and use it as the basis
% for your article
%
%%%%%%%%%%%%%%%%%%%%%%%% Springer-Verlag %%%%%%%%%%%%%%%%%%%%%%%%%%
%
\documentclass[epj]{svjour}
% Remove option referee for final version
%
% Remove any % below to load the required packages
%\usepackage{latexsym}
\usepackage{graphicx}
\usepackage{amsmath}
\usepackage{amssymb}
% etc
%
\begin{document}
%
\title{Electron kinetics across plasma-wall interfaces}
%\subtitle{Do you have a subtitle?\\ If so, write it here}
\author{P. Matthias \and R. Schneider \and G. Bandelow \and J. Duras \and K.-F.. L\"ueskow \and D. Kahnfeld \and S. Kemnitz \and L. Lewerentz \and P. Hacker
% \thanks is optional - remove next line if not needed
%\thanks{\emph{Present address:} Insert the address here if needed}%
}                     % Do not remove
%
%\offprints{}          % Insert a name or remove this line
%
\institute{Institut f\"ur Physik, Ernst-Moritz-Arndt-Universit\"at Greifswald, D-17489 Greifswald, Germany}
%
\date{Received: date / Revised version: date}
% The correct dates will be entered by Springer
%
\abstract{
The most fundamental response of an ionized gas to a macroscopic object 
is the formation of the plasma sheath. It is an electron depleted space 
charge region adjacent to the object build-up to screen the object's 
negative charge which arises from the accumulation of electrons from 
the plasma. The plasma sheath is thus the positively charged part of an 
electric double layer whose negatively charged part is inside the wall. 
In the course of the Transregional Collaborative Research Center SFB/TRR24 
we investigated, from a quantum-mechanical point of view, the elementary 
charge transfer processes responsible for the electric double layer 
and made first steps towards a description of the negative part of the 
layer inside the wall. Below we critically review our work, describe 
possible extensions, and identify key issues which need to be resolved 
to make further progress in the understanding of the electron kinetics 
across plasma-wall interfaces. In particular, we describe the methods 
we applied for the calculation of the electron sticking coefficient 
$S_e$, the ion-wall recombination probability $\alpha_w$, and the 
secondary electron emission coefficient $\gamma_e$ due specifically to 
charge-transferring atom-surface collisions. The concept of an electron
surface layer which we used to merge the sheath potential with the 
slowly varying potential inside the wall seen by surplus electrons
will be also reviewed as well as the approach we proposed to measure
the charge of dust grains optically by Mie scattering. Besides wall 
parameters, the microscopic look at sheath formation provides new vistas 
for the modeling and diagnostics of low-temperature gas discharges. 
A look into the future concludes therefore the presentation. 
%Each subsection contains therefore also a look into the future. 
%
\PACS{
      {52.40.Hf}{Plasma-material interaction, boundary layer effects}\and
      {52.40.Kh}{Plasma sheaths}\and
      {68.49.Jk}{Electron scattering from surfaces}\and
      {68.49.Sf}{Ion scattering from surfaces (charge transfer,sputtering,SIMS)}
     } % end of PACS codes
} %end of abstract
%
\maketitle
%

\begin{acknowledgement}
Support from the Deutsche Forschungsgemeinschaft through Project No. B10 
of the Transregional Collaborative Research Center SFB/TRR 24 is greatly 
acknowledged. 
\end{acknowledgement}


\chapter{Introduction}

Capacitive-coupled discharges with a radio-frequency modulated voltage are important for plasma etching \cite{etching} and sputter processes \cite{sputter}.

The latest results of an experiment with electronegative discharges have shown a high-energy peak of the negative ions arriving at the anode, depending on the cathode material used. 
One possible explanation is ionization at or close to the surface for the production of negative ions.

\begin{figure}[htbp]
\centering
    \includegraphics[width=0.8\textwidth]{../pics2/intro_dist.png}
    \caption{Experimental measured negatively charged oxygen ions arriving at the grounded anode as shown in \cite{basti}. The cathode material used was Magnesiumoxide (MgO) and the cathode was powered with $50\ein{W}$.}
    \label{fig:intro}
\end{figure}

I want to improve the insight into microscopic mechanisms of electronegative plasmas and surface ionization effects using a MCC-PIC simulation.



\section{Surface effects and secondary ion emission}

In comparison to the known secondary electron emission there also exists \textbf{s}econdary \textbf{i}on \textbf{e}mission (SIE).
Theoretical studies of surface ionization are mostly devoted to the production of positive ions from incident atoms having thermal energies \cite{surf_ioniz}. 
The degree of ionization can be derived by applying thermodynamic arguments.
The ionization coefficient of M is then given by
\begin{align}
    \alpha^+(M^+) &= \frac{n^+}{n} \nonumber\\
                  &= \frac{1-r^+}{1-r} \cdot \frac{w^+}{w} \mathrm{exp}\left(
                    \frac{\bar{\Phi}^+ + e \sqrt{eF} - I(M)}{k_B T} \right) \; ,
    \label{eq:pos_inj}
\end{align}
where $n^+$ and $n$ are the numbers of $M^+$ and $M$ coming from unit surface area per unit time, $w^+/w$ is the statistical weight ratio of $M^+$ to $M$, $r^+$ and $r$ are the internal reflection coefficients at the potential barrier on the emitter surface, $\bar{\Phi}^+$ is the average work function,  $T$ is the absolute temperature at the surface, $F$ is the externally applied field and $I(M)$ is the initial energy of the impinging atoms. 
 
With little adaptations I can transfer equation (\ref{eq:pos_inj}) for treating a surface which emits negative ions.
I assume a negatively biased surface like the cathode in an asymmetric ccrf discharge where the equilibrium
\begin{align}
    X + e^- \; \mathrm{(in\; the\; substrate\; metal)} \leftrightharpoons X^-
\end{align}
is attained at the surface.
Hence, the equation for the negative ionization coefficient follows in analogy to equation (\ref{eq:pos_inj})
\begin{align}
    \alpha^-(X^-) &= \frac{n^-}{n} \nonumber\\
                  &= \frac{1-r^-}{1-r} \cdot \frac{w^-}{w} \mathrm{exp}\left( 
                    \frac{-\bar{\Phi}^- + e \sqrt{eF} + A(X)}{k_B T} \right) \; ,
    \label{eq:neg_inj}
\end{align}
where $A(X)$ is the electron affinity of atom $X$, $\bar{\Phi}^-$ is the average work function effective for producing the negative ion $X^-$ on the metal surface and the other parameter are in analogy to the positive ion injection.
To the best of my knowledge. there are no detailed theoretical and experimental studies of the reflection coefficients $r$ for negative ions. \\ 
Therefore no detailed information about the cross-sections for such processes exists.
In this work I assume that negative ions are produced by a positive ion beam on the surface with an efficiency of $\eta=n_-/n_+$.\\ 
The additional flux of negatively charged particles leads to a reduction of the potential drop as in the case of secondary electron emission.

To get a microscopic knowledge of a ccrf discharge the discharge is simulated.
Since nonthermal low pressure plasmas are not collision dominated the particles are not Maxwell distributed.
The mean free paths are of the same magnitude as the electrode gap.
This means that fluid models fail at this point and kinetic models have to be applied.
I will use the PIC-MCC method which is widely used for plasma simulation.




\section{PIC-MCC}

\subsection{1d3v PIC}
The discharge of the experiment has a cylindrical geometry.
In the centre of the discharge the plasma does not see the edge of the electrodes and therefore no asymmetry effects may take place. 
For the middle of the discharge it is sufficient to do an one-dimensional approach.
The disadvantage of the 1d3v simulation is the neglect of different transport processes in radial direction. 
This leads to a discrepancy between simulation and real experiment.
I adjust my parameters in a way so that I receive comparable results close to one electrode, where the experimental measurements took place.
In this case I cannot compare absolute values but can still see trends in the distribution functions.\\
I can compare these results with a 2d3v simulation.
This will separate the asymmetric influence.


\chapter{PIC simulation of a ccrf discharge}
\label{PIC_discharge}

The system starts with a pressure of $10\ein{Pa}$ and a peak-to-peak voltage of $800\ein{V}$. 
From there on the pressure will be reduced to get to low pressures around $2\ein{Pa}$.
The radio-frequency is set to $13.56\ein{MHz}$.
The initial values for the considered electron density $n_e=5\cdot 10^{9}\ein{/cm^{-3}}$ and electron temperature $T_e = 4 \ein{eV}$ are set.\\
The Debye-length of the system is $\lambda_{Db} \approx 0.021 \ein{cm}$ and the electron plasma frequency is $\omega_{pe}\approx 3.99 \cdot 10^9 \ein{s^{-1}}$.
The electrode gap of the experiment is $5\ein{cm}$. 
To obtain a comparable plasma in 1D the simulation was done with a domain length of $6.72\ein{cm}$.\\
The results for low pressure discharges do not give us an explanation for the measured high-energy peak (s. figure \ref{fig:MgO}).

\newpage
\section{Discharge with secondary ion emission}

After I implemented the proposed SIE injection model of oxygen anions, one can see in figure ($\ref{fig:sims1}$) that the number density of the anions is higher and slightly shifted towards the cathode.
Here, I chose the injection coefficient $\eta=0.03$.

\begin{figure}[htbp]
    \centering
    \includegraphics[width=0.8\textwidth]{../pics2/5Pa_400V_density_sims.png}
    \caption{Density distribution of $e^-$, $O_2^+$ and $O^-$ with secondary ion emission at the cathode ($\eta=0.03$). The pressure was $5\ein{Pa}$ and the rf power was set to $U_{rf}=800\ein{V_{pp}}$. The arrow marks the little density peak of $O^-_s$ at the cathode sheath edge.}
    \label{fig:sims1}
\end{figure}

Since I want to study the behavior of the anions coming from the surface I separated the two species.
I refer to them as anions produced by volumetric processes in the plasma $O^-_p$ and surface anions $O^-_s$. 
%In figure (\ref{fig:sims2}) the two anion species number densities are shown.

%\begin{figure}[htbp]
%    \centering
%    \includegraphics[width=0.7\textwidth]{../pics/good/5Pa_400V_dens2_sims.png}
%    \caption{Same density profile as in figure (\ref{fig:sims1}) with the number densities of $O^-_p$ and $O^-_s$ separated.}
%    \label{fig:sims2}
%\end{figure}

It is visible that the density distribution of $O^-_s$ is of one magnitude higher than $O^-_p$. 
This depends on the chosen injection ratio $\eta$. 
The surface ion density has a peak close to the cathode resulting from the injection of $O^-_s$. 
In addition a small density peak at the sheath edge in front of the cathode is noticeable. 
It forms due to elastic collisions of the anions $O^-_s$ in the sheath. \\
The $O^-_s$ get accelerated in the sehat, cross the bulk and then get reflected in the anode sheath similar to electrons.\\
\begin{figure}[htbp]
    \centering
    \includegraphics[width=0.8\textwidth]{../pics2/5Pa_400V_energy_sims.png}
    \caption{Energy distribution of $O⁻_s$ at $5\ein{Pa}$.}
    %The algebraic sign in front of the energy symbolizes the direction where the ions are heading (negative towards the anode, positive towards the cathode).
    \label{fig:sims_distribution_top}
\end{figure}
With additional SIE a high-energy peak builds up. 
It decays with the time of flight (distance to the cathode) due to charge-exchange and elastic collisions with neutral molecules $O_2$ which results in an energy loss for the anion.
Also a part of the anions gets detached by neutrals or recombines with positive ions.
Until now it was assumed that if an anion collides with a neutral it almost gets detached every time.
But the anions also do elastic collisions which leads to an energy loss and a plateau in the energy. 
In figure (\ref{fig:coll_sims}) the difference between the numbers of elastic collisions for a normal discharge and a discharge with additional SIE are shown. 
\begin{figure}[htbp]
    \centering
    \subfloat[][]{  \includegraphics[width=0.4\textwidth]{../pics2/elastic_coll.png}}
    \subfloat[][]{\includegraphics[width=0.4\textwidth]{../pics2/elastic_coll_sims.png}}
    \caption{Number of elastic collisions of negative ions $O^-$ with neutral molecules  $O_2$ per $10^5$ steps. 
    The figure shows them in simulation without SIE (a) and with SIE ($\eta=0.03$) (b) where between the two $O^-$ species is differentiated.}
    \label{fig:coll_sims}
\end{figure}
Most elastic collisions happen in the bulk while the sheaths are mostly collisionless.
But for the surface ions $O^-_s$ one can see that the collisions in the cathode sheath cannot be neglected.
They lead to an energy loss for the anions which impacts on their energy distribution.\\
In figure (\ref{fig:sims_distribution_top}) a structure in the lower energy area can be seen.
The performed elastic collisions during this phase lead to a peak structure in the ion energy distribution.
To get an impression of the dynamics in the sheath the phase resolved energy distribution for $O^-_{s}$ is shown in figure (\ref{fig:phase_resolved})). \\
\begin{figure}[htbp]
    \centering
    \includegraphics[width=0.8\textwidth]{../pics2/time_solo.png}
    \caption{Same energy distribution as in figure (\ref{fig:sims_distribution_top}) at t=0 of the rf cycle, which is equal to $U(t)=0$.}
    \label{fig:phase_resolved_solo}
\end{figure}
The density peak at the sheath edge (as seen in figure (\ref{fig:sims1})) originates from the low-energy peak in the energy distribution.
With the approximate transit time of an anion $\tau_{ion}$ and the rf-cycle time $\tau_{rf}$ one can calculate the transit time $\tau_{ion}$.
Assuming an average ion energy of $40-50 \ein{eV}$ and a traveled distance of $\approx 1\ein{cm}$ it follows the ratio $\frac{\tau_{ion}}{\tau_{rf}}\approx 4.5$.
This is the number of rf cycles an anion stays in the sheath.
Hence the number of peaks in the ion energy distribution must be similar.
In figure (\ref{fig:sims_distribution}) one can see a high-energy peak and 4-5 low-energy peaks in the bulk region. 
At the sheath edge these density waves overlay.
Hence, the energy plateau of the anions is mainly influenced by elastic collisions.
This explains the density peak at the sheath edge.
%There may be other structures at lowest energies too, but they are not distinguishable from the cold anions in the bulk.
In the experiment where the cathode potential is shifted by the self-bias voltage the resulting potential is asymmetric. 
That means, the anions can get enough energy to get to the anode while in the current simulation due to the forced symmetry they get reflected by the sheath potential.
\begin{figure}[htbp]
        \centering
    \includegraphics[width=0.8\textwidth]{../pics2/neg_mg_one_only.png}
    \centering
    \includegraphics[width=0.8\textwidth]{../pics2/5Pa_400V_Om_energy_cuts.png}
    \caption{Energy distribution of negative ions $O^-$. Top: Experimental results for $MgO$ measured at the anode for different rf powers. Bottom: Simulation result with 1d3v PIC simulation with additional SIE taken at the anode sheath edge.}
    \label{fig:compare_ied}
\end{figure}
Figure (\ref{fig:compare_ied}) confirms that negative ions produced at the surface may lead to the measured high-energy peak. 
But the energy distribution function of the simulation has additional low energy peaks (at $< 100\ein{eV}$), too.
I consider them to be created due to the lack of asymmetry in the simulation.
In the experiment all high-energy anions are detected and thereby removed of the discharge.\\
Additional studies have been done considering variation of pressure, voltage and injection coefficient.
These studies all support the alredy proposed thesis.



\chapter{PIC simulation of a ccrf discharge}
\label{PIC_discharge}

The system starts with a pressure of $10\ein{Pa}$ and a peak-to-peak voltage of $800\ein{V}$. 
From there on the pressure will be reduced to get to low pressures around $2\ein{Pa}$.
The radio-frequency is set to $13.56\ein{MHz}$.
The initial values for the considered electron density $n_e=5\cdot 10^{9}\ein{/cm^{-3}}$ and electron temperature $T_e = 4 \ein{eV}$ are set.\\
The Debye-length of the system is $\lambda_{Db} \approx 0.021 \ein{cm}$ and the electron plasma frequency is $\omega_{pe}\approx 3.99 \cdot 10^9 \ein{s^{-1}}$.
The electrode gap of the experiment is $5\ein{cm}$. 
To obtain a comparable plasma in 1D the simulation was done with a domain length of $6.72\ein{cm}$.\\
The results for low pressure discharges do not give us an explanation for the measured high-energy peak (s. figure \ref{fig:MgO}).

\newpage
\section{Discharge with secondary ion emission}

After I implemented the proposed SIE injection model of oxygen anions, one can see in figure ($\ref{fig:sims1}$) that the number density of the anions is higher and slightly shifted towards the cathode.
Here, I chose the injection coefficient $\eta=0.03$.

\begin{figure}[htbp]
    \centering
    \includegraphics[width=0.8\textwidth]{../pics2/5Pa_400V_density_sims.png}
    \caption{Density distribution of $e^-$, $O_2^+$ and $O^-$ with secondary ion emission at the cathode ($\eta=0.03$). The pressure was $5\ein{Pa}$ and the rf power was set to $U_{rf}=800\ein{V_{pp}}$. The arrow marks the little density peak of $O^-_s$ at the cathode sheath edge.}
    \label{fig:sims1}
\end{figure}

Since I want to study the behavior of the anions coming from the surface I separated the two species.
I refer to them as anions produced by volumetric processes in the plasma $O^-_p$ and surface anions $O^-_s$. 
%In figure (\ref{fig:sims2}) the two anion species number densities are shown.

%\begin{figure}[htbp]
%    \centering
%    \includegraphics[width=0.7\textwidth]{../pics/good/5Pa_400V_dens2_sims.png}
%    \caption{Same density profile as in figure (\ref{fig:sims1}) with the number densities of $O^-_p$ and $O^-_s$ separated.}
%    \label{fig:sims2}
%\end{figure}

It is visible that the density distribution of $O^-_s$ is of one magnitude higher than $O^-_p$. 
This depends on the chosen injection ratio $\eta$. 
The surface ion density has a peak close to the cathode resulting from the injection of $O^-_s$. 
In addition a small density peak at the sheath edge in front of the cathode is noticeable. 
It forms due to elastic collisions of the anions $O^-_s$ in the sheath. \\
The $O^-_s$ get accelerated in the sehat, cross the bulk and then get reflected in the anode sheath similar to electrons.\\
\begin{figure}[htbp]
    \centering
    \includegraphics[width=0.8\textwidth]{../pics2/5Pa_400V_energy_sims.png}
    \caption{Energy distribution of $O⁻_s$ at $5\ein{Pa}$.}
    %The algebraic sign in front of the energy symbolizes the direction where the ions are heading (negative towards the anode, positive towards the cathode).
    \label{fig:sims_distribution_top}
\end{figure}
With additional SIE a high-energy peak builds up. 
It decays with the time of flight (distance to the cathode) due to charge-exchange and elastic collisions with neutral molecules $O_2$ which results in an energy loss for the anion.
Also a part of the anions gets detached by neutrals or recombines with positive ions.
Until now it was assumed that if an anion collides with a neutral it almost gets detached every time.
But the anions also do elastic collisions which leads to an energy loss and a plateau in the energy. 
In figure (\ref{fig:coll_sims}) the difference between the numbers of elastic collisions for a normal discharge and a discharge with additional SIE are shown. 
\begin{figure}[htbp]
    \centering
    \subfloat[][]{  \includegraphics[width=0.4\textwidth]{../pics2/elastic_coll.png}}
    \subfloat[][]{\includegraphics[width=0.4\textwidth]{../pics2/elastic_coll_sims.png}}
    \caption{Number of elastic collisions of negative ions $O^-$ with neutral molecules  $O_2$ per $10^5$ steps. 
    The figure shows them in simulation without SIE (a) and with SIE ($\eta=0.03$) (b) where between the two $O^-$ species is differentiated.}
    \label{fig:coll_sims}
\end{figure}
Most elastic collisions happen in the bulk while the sheaths are mostly collisionless.
But for the surface ions $O^-_s$ one can see that the collisions in the cathode sheath cannot be neglected.
They lead to an energy loss for the anions which impacts on their energy distribution.\\
In figure (\ref{fig:sims_distribution_top}) a structure in the lower energy area can be seen.
The performed elastic collisions during this phase lead to a peak structure in the ion energy distribution.
To get an impression of the dynamics in the sheath the phase resolved energy distribution for $O^-_{s}$ is shown in figure (\ref{fig:phase_resolved})). \\
\begin{figure}[htbp]
    \centering
    \includegraphics[width=0.8\textwidth]{../pics2/time_solo.png}
    \caption{Same energy distribution as in figure (\ref{fig:sims_distribution_top}) at t=0 of the rf cycle, which is equal to $U(t)=0$.}
    \label{fig:phase_resolved_solo}
\end{figure}
The density peak at the sheath edge (as seen in figure (\ref{fig:sims1})) originates from the low-energy peak in the energy distribution.
With the approximate transit time of an anion $\tau_{ion}$ and the rf-cycle time $\tau_{rf}$ one can calculate the transit time $\tau_{ion}$.
Assuming an average ion energy of $40-50 \ein{eV}$ and a traveled distance of $\approx 1\ein{cm}$ it follows the ratio $\frac{\tau_{ion}}{\tau_{rf}}\approx 4.5$.
This is the number of rf cycles an anion stays in the sheath.
Hence the number of peaks in the ion energy distribution must be similar.
In figure (\ref{fig:sims_distribution}) one can see a high-energy peak and 4-5 low-energy peaks in the bulk region. 
At the sheath edge these density waves overlay.
Hence, the energy plateau of the anions is mainly influenced by elastic collisions.
This explains the density peak at the sheath edge.
%There may be other structures at lowest energies too, but they are not distinguishable from the cold anions in the bulk.
In the experiment where the cathode potential is shifted by the self-bias voltage the resulting potential is asymmetric. 
That means, the anions can get enough energy to get to the anode while in the current simulation due to the forced symmetry they get reflected by the sheath potential.
\begin{figure}[htbp]
        \centering
    \includegraphics[width=0.8\textwidth]{../pics2/neg_mg_one_only.png}
    \centering
    \includegraphics[width=0.8\textwidth]{../pics2/5Pa_400V_Om_energy_cuts.png}
    \caption{Energy distribution of negative ions $O^-$. Top: Experimental results for $MgO$ measured at the anode for different rf powers. Bottom: Simulation result with 1d3v PIC simulation with additional SIE taken at the anode sheath edge.}
    \label{fig:compare_ied}
\end{figure}
Figure (\ref{fig:compare_ied}) confirms that negative ions produced at the surface may lead to the measured high-energy peak. 
But the energy distribution function of the simulation has additional low energy peaks (at $< 100\ein{eV}$), too.
I consider them to be created due to the lack of asymmetry in the simulation.
In the experiment all high-energy anions are detected and thereby removed of the discharge.\\
Additional studies have been done considering variation of pressure, voltage and injection coefficient.
These studies all support the alredy proposed thesis.



\begin{figure}[htbp]
    \centering
    \includegraphics[width=0.7\textwidth]{../pics2/power_energy_cuts.png}
    \caption{$O_s^-$ energy distribution under different driver voltages $U_{rf}$ with $5\ein{Pa}$ and $\eta=0.01$.}
    \label{fig:voltage}
\end{figure}

\begin{figure}[htbp]
    \centering
    \includegraphics[width=0.7\textwidth]{../pics2/eta_energy_cuts.png}
    \caption{$O_s^-$ energy distribution for different $\eta$ with $5\ein{Pa}$ and $U_{rf}=800V_{pp}$.}
    \label{fig:eta}
\end{figure}


\begin{figure}[htbp]
    \centering
    \includegraphics[width=0.7\textwidth]{../pics2/pressure_energy_cuts.png}
    \caption{$O_s^-$ energy distribution under different pressures with $U_{pp}=\ein{V}$ and $\eta=0.01$.}
    \label{fig:pressure}
\end{figure}








%
\bibliographystyle{Style/icpig}
%\bibliographystyle{Style/tr24-ohne-titel}
\bibliography{MicroPhys}

\end{document}


