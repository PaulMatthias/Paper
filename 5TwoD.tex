\chapter{Two-Dimensional PIC simulation}
The two-dimensional PIC code used was developed by the group of Prof. Dr. Ralf Schneider.
Over several years the code was developed and optimized for ion thruster modeling.
I adapted it to simulate a capacitively coupled rf discharge.\\
Due to the symmetry of the discharge a cylindrical geometry in r and z direction is used.

\begin{figure}[htbp]
    \centering
    \includegraphics[width=0.7\textwidth]{../pics/2d_domain.png}
    \caption{Schematic figure of the geometry of the discharge. On the left hand side is shown the whole discharge in a cylinder, where the simulation domain is marked as a slice. On the right hand side the resulting domain with the corresponding boundary conditions is shown.}
    \label{fig:2d_domain}
\end{figure}

To realize an asymmetric discharge the size of the cathode is set smaller than the size of the anode.
This in addition to the grounded wall leads to a strong asymmetry.
In an asymmetric discharge the high mobility of the electrons charges the cathode, which then gets a negative self-bias voltage. 

\section{Simulation of an asymmetric ccrf discharge}
A bigger domain size (500x200), which gets closer to the experimental size, is used from now on.
In the following simulations with an asymmetry factor of 0.6 will be used, resulting in a cathode radius of $\approx2.5\ein{cm}$. 
In figure (\ref{fig:potential_argon1}) the averaged potential is shown.
Due to the asymmetry a negative self-bias voltage at the cathode builds up. 
This is a main difference to one-dimensional simulations.
To compare the potential with one-dimensional simulations (s. figure (\ref{fig:profile})) one has to look at the potential at the axis nr=0 (s. figure (\ref{fig:potential_argon2})).
The shape is the same despite the self-bias voltage which shows the advantage of two-dimensional simulations to obtain a realistic potential.

HERE PUT NEW PIC OF POTENTIAL!!!!!!!!!!!!!!!!!!!!!!!!!!!!!!!!!!!!!!!!!!!!!!!!!!!!!!!!!!!!!!!!!!!!!1
\begin{figure}[htbp]
    \centering
    \includegraphics[width=1\textwidth]{../pics2/argon_potential.png}
    \caption{Potential of an argon rf discharge with an asymmetry ratio of $0.6$. This leads to a cathode size of $r_{cathode}\approx2.5\ein{cm}$. The pressure was set to $30\ein{Pa}$ and the power was $U_{rf}=1000\ein{V_{pp}}$.}
    \label{fig:potential_argon1}
\end{figure}

\begin{figure}[htbp]
    \centering
    \includegraphics[width=0.7\textwidth]{../pics2/argon_potential_axis.png}
    \caption{Potential of an argon discharge at the axis $(r=0)$. }
    \label{fig:potential_argon2}
\end{figure}

In figure (\ref{fig:dens_argon}) the electron and ion number density after a runtime of 30 rf cycles after initialization are shown.
This is a not yet fully equilibrated state of the plasma, due to the long run-times.
One can see that the bulk region is slightly deformed at the cathode side.
This is due to the self-bias voltage which leads to a reduced electron flux towards the cathode.
The ions follow the motion of the electrons.

\begin{figure}[htbp]
    \centering
    \includegraphics[width=1\textwidth]{../pics2/argon_density.png}
    \caption{Charged particle density distribution of an argon discharge.}
    \label{fig:dens_argon}
\end{figure}
Now the density distribution can be compared with the previous one-dimensional results.
Therefore, the number density distributions of the charged species at the axis (s. figure (\ref{fig:dens_axis_argon})) are taken, which equals the one-dimensional domain.
\begin{figure}[htbp]
    \centering
    \includegraphics[width=0.7\textwidth]{../pics2/density_axis.png}
    \caption{Charged particle density distribution of an argon discharge at the center of the discharge from a two-dimensional PIC simulation at $30\ein{Pa}$ and $1000\ein{V_{pp}}$.}
    \label{fig:dens_axis_argon}
\end{figure}

\begin{figure}[htbp]
    \centering
    \includegraphics[width=0.7\textwidth]{../pics2/1d_argon.png}
    \caption{Charged particle density distribution of an argon discharge out of an one-dimensional PIC simulation at $30\ein{Pa}$ and $400\ein{V_{pp}}$.}
    \label{fig:dens_axis_argon}
\end{figure}
One can see the quasi-neutral bulk region and the ion flux towards the walls. 
There is a higher flux of ions towards the cathode than to the anode due to the self-bias voltage.
This additional flux is not covered by the one-dimensional simulation.
Additionally, the cathode sheath is larger in two dimensions, due to the self-bias voltage.
So in a one-dimensional simulation the total ion flux towards the cathode and the cathode sheath width are slightly underestimated.
Nevertheless, the form of the number density distributions is nearly the same, which shows that a one-dimensional simulation is a good approximation regarding the number densities for capacitively discharges near the center.\\
In addition to the density distributions and the potential a closer look at the energy distributions of the plasma particles is taken.
The ion energy distributions in r and z direction are shown in figure (\ref{fig:ion_energy_z}) and (\ref{fig:ion_energy_r}).
%I will not show the velocity distribution in $\theta$ direction, because there is no dynamic in this direction due to the symmetry of the discharge, so the particles will keep their initial thermal velocities.
%\begin{figure}[htbp]
%    \centering
%    \includegraphics[width=1\textwidth]{../pics2/500V_60Pa_velocity.png}
%    \caption{Argon ion velocity distribution in r and z direction. VELOCITY SCALING!!!!!!!!!!!!!!!!!!!!}
%    \label{fig:vel_argon}
%\end{figure}
\begin{figure}[htbp]
    \centering
    \includegraphics[width=1\textwidth]{../pics2/iDist_z.png}
    \caption{Ion energy distribution in z direction. (a) shows the averaged energy resolved for the vertical z axis and (b) for the radial r axis }
    \label{fig:ion_energy_z}
\end{figure}
The z component of the ion energy distribution shows the expected behavior of cold ions in the bulk and accelerating ions towards the electrodes (s. figure (\ref{fig:ion_energy_z} (a))), when averaged along r at fixed z.
%When the ion velocity distribution is averaged along z at fixed r (s. figure (\ref{fig:ion_energy_z} (b) )), 
The radial component of the ion energy distribution (s. figure (\ref{fig:ion_energy_r})) shows the same acceleration of the ions towards the wall.
%This will show up in the ion energy distribution function because the ions at the edge of the cathode will have more energy then in the center of the cathode.
\begin{figure}[htbp]
    \centering
    \includegraphics[width=1\textwidth]{../pics2/iDist_r.png}
    \caption{Ion energy distribution in r direction. (a) shows the averaged energy resolved for the vertical z axis and (b) for the radial r axis }
    \label{fig:ion_energy_r}
\end{figure}
Near the cathode edge (r=120) the ions are accelerated towards the cathode due to the radial potential gradient, which results in a larger ion energy.\\
When comparing the electron energy distributions with the results from the one dimensional PIC simulation (s. figure (\ref{fig:electron_energy})),
one can see that for the z component of the electron energy distribution (s. figure (\ref{fig:el_energy_z} (a))) the shape of the distributions is nearly the same, except near the cathode.
\begin{figure}[htbp]
    \centering
    \includegraphics[width=1\textwidth]{../pics2/eDist_z.png}
    \caption{Electron energy distribution in z direction. (a) shows the averaged energy resolved for the vertical z axis and (b) for the radial r axis }
    \label{fig:el_energy_z}
\end{figure}
Due to the asymmetry the potential has additional radial gradients, especially above the cathode (s. figure (\ref{fig:potential_argon1})).
This leads to a shift in the electron energy distribution function near the cathode, where the electrons get pushed in radial direction. 
In the experiment the cathode usually is not located at the wall, but stands in free space.
Here the cathode is embedded in a grounded ring.
So this is an effect which can be modified by moving the cathode into the domain.
Then the mesh and the boundary conditions have to be applied accordingly.
But I am not doing this approach in this thesis.\\
Naturally, the r component of the electron distribution function is affected by the same radial gradient of the potential and builds up an analogue distribution function (s. figure (\ref{fig:el_energy_r})).

\begin{figure}[htbp]
    \centering
    \includegraphics[width=1\textwidth]{../pics2/eDist_r.png}
    \caption{Electron energy distribution in r direction. (a) shows the averaged energy resolved for the vertical z axis and (b) for the radial r axis }
    \label{fig:el_energy_r}
\end{figure}

%Now, the former model of SIE is applied additionally.
%Since ordinary argon plasma does not contain negative ions, the injected anions are treated as negatively charged argon ions which do not collide with other particles.
%Following the argumentation of the one-dimensional model the injected anions should not be able to stay in the discharge, due to the additional energy from the self-bias voltage.
%The resulting energy distribution of the anions is shown in figure (\ref{fig:sie_twod}).
%\begin{figure}[htbp]
%    \centering
%    \includegraphics[width=1\textwidth]{../pics2/inDist_r.png}
%    \caption{Surface anion energy distribution in r direction. (a) shows the averaged energy resolved for the vertical z axis and (b) for the radial r axis }
%    \label{fig:el_energy_r}
%\end{figure}
%\begin{figure}[htbp]
%    \centering
%    \includegraphics[width=1\textwidth]{../pics2/inDist_z.png}
%    \caption{Surface anion energy distribution in z direction. (a) shows the averaged energy resolved for the vertical z axis and (b) for the radial r axis }
%    \label{fig:el_energy_r}
%\end{figure}
%As one can see the anions impinge on the anode and on the wall, since their energy is higher than the potential drop in front of them.
%This explains the discrepancy between the one-dimensional model and the experiment.\\

%Übergang

Now that I achieved a stable argon discharge in a two dimensional PIC simulation the next step would be to simulate an oxygen plasma and apply the proposed SIE processes.
The runtime for in depth studies of even larger domains exceeds the time limit of this thesis.
%Then i could run simulations with and without SIE and compare the results with the one dimensional simulations and i could get an even better comparison with the experiment.\\
But the two dimensional system is prepared for capacitively coupled rf plasmas and future research may continue this approach. \\
