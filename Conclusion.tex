\chapter{Conclusion}

In this thesis I successfully simulated electronegative capacitively coupled rf plasmas in one dimension under low pressure oxygen using the Particle-in-Cell method. 
Due to latest experimental results, showing a high-energy peak in the anion energy distribution depending on the used cathode material, secondary ion emission effects were additionally applied.
Since the theoretical background of these surface effects is not given yet, an injection mechanism depending on impinging ions was implemented.
With the model of SIE I was able to recreate and explain qualitatively the experimental measured high-energy peak of negative oxygen ions $O^-$ at the anode.
The simulation results yielded an additional low-energy peak in contradiction to the experiment.
It could be shown that the peak forms due to the surface anions staying in the discharge and then colliding with neutrals in the sheaths.
I then continued and varied the system parameters to get a quantitative understanding of the anion energy distribution function in respect to applied power, pressure and injection ratio $\eta$.
The acquired results were very close to the experimental measurements except for the variation of the injection parameter $\eta$, which only slightly affected the resulting anion energy distributions.
It was shown that due to the lack of asymmetry, which cannot be covered in an one-dimensional simulation, the injected anions stay in the discharge and then add up to a constant state, not influenced by the different chosen injection ratios $\eta$.
The experiment otherwise states big differences for different cathode materials.\\
Therefore, a two-dimensional model was introduced, but the corresponding research is still at the beginning.
With this two-dimensional model a micro discharge and an asymmetric ccrf discharge with argon as a working gas have been simulated.
Especially, the influence of the asymmetry on the number density distribution and on the energy distribution could be observed.
Through comparison it was shown that the one-dimensional model is a good approximation for the center of the discharge, except for the forming of a self-bias voltage.
%To prove the assumptions made in the one-dimensional simulation regarding the negative anions, SIE has been implemented in the tow-dimensional model, too.
%The injected anions impinged on the anode, where they were absorbed and hence removed from the system.
%This explained the discrepancy observed in the former model.
Due to runtime reasons an in depth study lays beyond the scope of this thesis.\\
The next step would be to implement oxygen as a working gas and apply physically motivated SIE in the two dimensional model.
Then, one could apply different cathode materials to see the influence of different work functions on the anion $O^-$ energy distribution function, which could not be obtained in one dimension.\\
To the best of my knowledge until now there do not exist any publications of two dimensional PIC simulations of low temperature laboratory plasmas yet.
So the developed tool has still a lot of potential.\\



