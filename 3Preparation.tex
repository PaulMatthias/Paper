\chapter{PIC simulation of a ccrf discharge}
\label{PIC_discharge}

The system starts with a pressure of $10\ein{Pa}$ and a peak-to-peak voltage of $800\ein{V}$. 
From there on the pressure will be reduced to get to low pressures around $2\ein{Pa}$.
The radio-frequency is set to $13.56\ein{MHz}$.
The initial values for the considered electron density $n_e=5\cdot 10^{9}\ein{/cm^{-3}}$ and electron temperature $T_e = 4 \ein{eV}$ are set.\\
The Debye-length of the system is $\lambda_{Db} \approx 0.021 \ein{cm}$ and the electron plasma frequency is $\omega_{pe}\approx 3.99 \cdot 10^9 \ein{s^{-1}}$.
The electrode gap of the experiment is $5\ein{cm}$. 
To obtain a comparable plasma in 1D the simulation was done with a domain length of $6.72\ein{cm}$.\\
The results for low pressure discharges do not give us an explanation for the measured high-energy peak (s. figure \ref{fig:MgO}).

\newpage
\section{Discharge with secondary ion emission}

After I implemented the proposed SIE injection model of oxygen anions, one can see in figure ($\ref{fig:sims1}$) that the number density of the anions is higher and slightly shifted towards the cathode.
Here, I chose the injection coefficient $\eta=0.03$.

\begin{figure}[htbp]
    \centering
    \includegraphics[width=0.8\textwidth]{../pics2/5Pa_400V_density_sims.png}
    \caption{Density distribution of $e^-$, $O_2^+$ and $O^-$ with secondary ion emission at the cathode ($\eta=0.03$). The pressure was $5\ein{Pa}$ and the rf power was set to $U_{rf}=800\ein{V_{pp}}$. The arrow marks the little density peak of $O^-_s$ at the cathode sheath edge.}
    \label{fig:sims1}
\end{figure}

Since I want to study the behavior of the anions coming from the surface I separated the two species.
I refer to them as anions produced by volumetric processes in the plasma $O^-_p$ and surface anions $O^-_s$. 
%In figure (\ref{fig:sims2}) the two anion species number densities are shown.

%\begin{figure}[htbp]
%    \centering
%    \includegraphics[width=0.7\textwidth]{../pics/good/5Pa_400V_dens2_sims.png}
%    \caption{Same density profile as in figure (\ref{fig:sims1}) with the number densities of $O^-_p$ and $O^-_s$ separated.}
%    \label{fig:sims2}
%\end{figure}

It is visible that the density distribution of $O^-_s$ is of one magnitude higher than $O^-_p$. 
This depends on the chosen injection ratio $\eta$. 
The surface ion density has a peak close to the cathode resulting from the injection of $O^-_s$. 
In addition a small density peak at the sheath edge in front of the cathode is noticeable. 
It forms due to elastic collisions of the anions $O^-_s$ in the sheath. \\
The $O^-_s$ get accelerated in the sehat, cross the bulk and then get reflected in the anode sheath similar to electrons.\\
\begin{figure}[htbp]
    \centering
    \includegraphics[width=0.8\textwidth]{../pics2/5Pa_400V_energy_sims.png}
    \caption{Energy distribution of $O⁻_s$ at $5\ein{Pa}$.}
    %The algebraic sign in front of the energy symbolizes the direction where the ions are heading (negative towards the anode, positive towards the cathode).
    \label{fig:sims_distribution_top}
\end{figure}
With additional SIE a high-energy peak builds up. 
It decays with the time of flight (distance to the cathode) due to charge-exchange and elastic collisions with neutral molecules $O_2$ which results in an energy loss for the anion.
Also a part of the anions gets detached by neutrals or recombines with positive ions.
Until now it was assumed that if an anion collides with a neutral it almost gets detached every time.
But the anions also do elastic collisions which leads to an energy loss and a plateau in the energy. 
In figure (\ref{fig:coll_sims}) the difference between the numbers of elastic collisions for a normal discharge and a discharge with additional SIE are shown. 
\begin{figure}[htbp]
    \centering
    \subfloat[][]{  \includegraphics[width=0.4\textwidth]{../pics2/elastic_coll.png}}
    \subfloat[][]{\includegraphics[width=0.4\textwidth]{../pics2/elastic_coll_sims.png}}
    \caption{Number of elastic collisions of negative ions $O^-$ with neutral molecules  $O_2$ per $10^5$ steps. 
    The figure shows them in simulation without SIE (a) and with SIE ($\eta=0.03$) (b) where between the two $O^-$ species is differentiated.}
    \label{fig:coll_sims}
\end{figure}
Most elastic collisions happen in the bulk while the sheaths are mostly collisionless.
But for the surface ions $O^-_s$ one can see that the collisions in the cathode sheath cannot be neglected.
They lead to an energy loss for the anions which impacts on their energy distribution.\\
In figure (\ref{fig:sims_distribution_top}) a structure in the lower energy area can be seen.
The performed elastic collisions during this phase lead to a peak structure in the ion energy distribution.
To get an impression of the dynamics in the sheath the phase resolved energy distribution for $O^-_{s}$ is shown in figure (\ref{fig:phase_resolved})). \\
\begin{figure}[htbp]
    \centering
    \includegraphics[width=0.8\textwidth]{../pics2/time_solo.png}
    \caption{Same energy distribution as in figure (\ref{fig:sims_distribution_top}) at t=0 of the rf cycle, which is equal to $U(t)=0$.}
    \label{fig:phase_resolved_solo}
\end{figure}
The density peak at the sheath edge (as seen in figure (\ref{fig:sims1})) originates from the low-energy peak in the energy distribution.
With the approximate transit time of an anion $\tau_{ion}$ and the rf-cycle time $\tau_{rf}$ one can calculate the transit time $\tau_{ion}$.
Assuming an average ion energy of $40-50 \ein{eV}$ and a traveled distance of $\approx 1\ein{cm}$ it follows the ratio $\frac{\tau_{ion}}{\tau_{rf}}\approx 4.5$.
This is the number of rf cycles an anion stays in the sheath.
Hence the number of peaks in the ion energy distribution must be similar.
In figure (\ref{fig:sims_distribution}) one can see a high-energy peak and 4-5 low-energy peaks in the bulk region. 
At the sheath edge these density waves overlay.
Hence, the energy plateau of the anions is mainly influenced by elastic collisions.
This explains the density peak at the sheath edge.
%There may be other structures at lowest energies too, but they are not distinguishable from the cold anions in the bulk.
In the experiment where the cathode potential is shifted by the self-bias voltage the resulting potential is asymmetric. 
That means, the anions can get enough energy to get to the anode while in the current simulation due to the forced symmetry they get reflected by the sheath potential.
\begin{figure}[htbp]
        \centering
    \includegraphics[width=0.8\textwidth]{../pics2/neg_mg_one_only.png}
    \centering
    \includegraphics[width=0.8\textwidth]{../pics2/5Pa_400V_Om_energy_cuts.png}
    \caption{Energy distribution of negative ions $O^-$. Top: Experimental results for $MgO$ measured at the anode for different rf powers. Bottom: Simulation result with 1d3v PIC simulation with additional SIE taken at the anode sheath edge.}
    \label{fig:compare_ied}
\end{figure}
Figure (\ref{fig:compare_ied}) confirms that negative ions produced at the surface may lead to the measured high-energy peak. 
But the energy distribution function of the simulation has additional low energy peaks (at $< 100\ein{eV}$), too.
I consider them to be created due to the lack of asymmetry in the simulation.
In the experiment all high-energy anions are detected and thereby removed of the discharge.\\
Additional studies have been done considering variation of pressure, voltage and injection coefficient.
These studies all support the alredy proposed thesis.


