
\chapter{Introduction}

Capacitive-coupled discharges with a radio-frequency modulated voltage are important for plasma etching \cite{etching} and sputter processes \cite{sputter}.

The latest results of an experiment with electronegative discharges have shown a high-energy peak of the negative ions arriving at the anode, depending on the cathode material used. 
One possible explanation is ionization at or close to the surface for the production of negative ions.

\begin{figure}[htbp]
\centering
    \includegraphics[width=0.8\textwidth]{../pics2/intro_dist.png}
    \caption{Experimental measured negatively charged oxygen ions arriving at the grounded anode as shown in \cite{basti}. The cathode material used was Magnesiumoxide (MgO) and the cathode was powered with $50\ein{W}$.}
    \label{fig:intro}
\end{figure}

I want to improve the insight into microscopic mechanisms of electronegative plasmas and surface ionization effects using a MCC-PIC simulation.


