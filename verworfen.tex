When the plasma is equilibrated the different species have a Maxwell distribution

\begin{align}
    f_s(\vec{v}) = \left(\frac{m_s}{2 \pi k_B T_s}\right)^{\frac{3}{2}} \mathrm{exp}\left\{ \frac{-m_s v^2}{2 k_B T_s}\right\} \; .
\end{align}

In respect to the value of the velocity $|\vec{v}|=v$ the distribution changes to 

\begin{align}
    f_s(v) = 4\pi v^2 \left(\frac{m_s}{2 \pi k_B T_s}\right)^{\frac{3}{2}} \mathrm{exp}\left\{ \frac{-m_s v^2}{2 k_B T_s}\right\} \; ,
\end{align}

with the average velocity value 

\begin{align}
\langle v_s \rangle = \sqrt{\frac{8 k_B T_s}{\pi m_s}}
\end{align}

and the root-mean-square velocity

\begin{align}
    \sqrt{\langle v_s^2 \rangle} = \sqrt{\frac{k_B T_s}{\pi m_s}} 
\end{align}

($k_B$ Boltzmann constant). 

------------------------------------------------------------------


In this thesis we won't consider magnetic fields so we neglect them. The magnetic fields produced by moving charges are neglected as well.\\


------------------------------------------------------------------------------

Another source of charged particles is the ion emission of the wall then hit by an ion. 
When a positive ion hits the wall it may emit a negative ion depending on its material. 
The ratio of impinging ions and emitted negative ions is further named as $\eta$. 
The production of negative ions due to plasma wall interaction is a current topic.
So the theoretical background for this process is not given yet.
In this thesis we will stick to the experiment and try to simulate naive surface processes.
We will see that we can simulate a plasma which energy distribution functions are close to  experimental results by injecting negative ions at the powered cathode.\\
Future studies will have to show what processes lead to this injection.



\section{Electronegative plasmas}

\subsection{Characteristics}

In comparison to ordinary plasma (i. e. Argon or Xenon) which consist of electrons, positive ions and neutrals, electronegative plasma also contains 
negative ions. 
They are of high importance for plasma-assisted material processing \cite{sputter}. 
The majority of negative ions formed in low pressure plasmas are by dissociative attachment of molecular species
\begin{align}
    e^- + \mathrm{AB} \rightarrow \mathrm{A} + \mathrm{B}^- \; .
\end{align}
When an electron attaches to a neutral molecule it gets excited.
It is either excited to a real or virtual state which has a curve crossing with a dissociative state (s. figure \ref{fig:ass_det}).
The fragments my be produced with significant kinetic energy.

\begin{figure}[htbp]
    \includegraphics[width=.7\textwidth]{../pics/ass_det.eps}
    \caption{Dissociation attachment of a molecule AB attached with an electron $e^-$ into A and $\mathrm{B}^-$.}
    \label{fig:ass_det}
\end{figure}

Another channel for negative ion production is through three-body non-dissociative attachment.
But further we are only concentrating on the ions produced by the dissociative attachment
because their production rate is several magnitudes larger in our referred system. \\
Negative ions can get lost through multiple reactions.
In our model we will consider neutral detachment, impact detachment and neutralization 

\begin{align}
    \mathrm{neutral\, detachment} \; A^- + B &\rightarrow A + B + e^- \nonumber\; , \\
    \mathrm{impact\, detachment} \; A^- + e^- &\rightarrow A + 2e^-  \; ,\\
    \mathrm{neutralization} \; A^- + B^+ &\rightarrow A + B \nonumber \; . 
\end{align}
The responding cross sections can be found in \ref{cross-section}. 
We will later see that most of the ions get lost through neutral detachment and neutralization.\\
As with positive ions we also calculate charge exchange collisions with the negative ions

\begin{align}
    \mathrm{charge\, exchange} \; A + B^- \rightarrow A^- + B  \; ,
\end{align}

which becomes important when we introduce surface effects.
The fast ions coming from the surface will cool down through charge exchange collisions.\\
Considering the additional reaction channels (in our case with Oxygen as the operating gas) we get our rate equations

\begin{align}
    \frac{\mathrm{d}n_e}{\mathrm{dt}}&=n_e O_2 k_1 + O^- O_2 k_1 - n_e O^- k_2 - n_e O_2^+ k_2 - n_e O_2 k_2 - \nabla \vec{\Phi}_e \nonumber \; ,\\
    \frac{\mathrm{d}O_2^+}{\mathrm{dt}}&=n_e O_2 k_1 - n_e O_2^+ k_2 - O^- O_2^+ k_2 - \nabla \vec{\Phi}_{O_2^+} \nonumber \; , \\
    \frac{\mathrm{d}O^-}{\mathrm{dt}}&=n_e O_2 k_1 - O^- O_2^+ k_2 - O^- O_2 k_2 - \nabla \vec{\Phi}_{O^-} \nonumber \; ,\\
    \vec{\Phi}_i&= q_i\mu_iN_i\vec{E}-D_i\nabla N_i \nonumber \qquad \mathrm{and} \\
    \nabla \vec{E}&= \frac{q}{\epsilon_0}\left( O_2^+ - O^- - n_e\right) \; .
\label{rates}
\end{align}

In principle negative ions behave like heavy, cold electrons ($T_i<<T_e$).
They obey the same kinetic and transport laws.
For conventional low pressure plasmas negative ions are confined by the plasma potential.
The main loss channel is hence by volumetric processes.
This leads to a density distribution (\ref{fig:profile}) with an electronegative core and regions where negative ions are excluded \cite{profile}.\\

\begin{figure}[htbp]
\centering
\includegraphics[width=0.8\textwidth]{../pics/5Pa_800V_profile.eps}
\caption{1D PIC simulation of Oxygen plasma showing plasma potential, density distributions and quasi neutrality.}
\label{fig:profile}
\end{figure}



One kind of electronegative plasmas is an Oxygen plasma which plays an important role for etching and thin-film deposition techniques \cite{etching}.\\
The existence of negative ions has a huge impact on the plasma characteristics, as it changes the ion density distribution which may lead to a central quasi-neutral ion-ion plasma and a peripheral electron-ion plasma.  
The electronegativity is described by the ratio of negative ions to electron $n_-/n_e$.
We talk about electronegative plasma when this ratio is $n_-/n_e>1$. \\
The electronegativity may lead to different kinds of instabilities.
At this point we only mention the attachment-induced ionization instability in oxygen plasmas \cite{instability}.
A small initial increase of the electron density leads to a shift of the equilibrium state between ionization and electron attachment processes. 
Under special conditions an additional increase of the electron density via a decrease in electron temperature is possible. 
The plasma becomes unstable and electron density fluctuations take place. 
These fluctuations are equivalent to fluctuations of the negative ion density because of the strong coupling between the electron density and the attachment process. 
But this instability is only to happen at high pressures (>30$\ein{Pa}$) and under certain rf powers.\\

-------------------------------------------------------------------------------------------

We focus especially onto a capacitive-coupled asymmetric rf-discharge in low pressure Oxygen (<10$\ein{Pa}$). 
That ensures us mostly stable discharges.
The experimental setup is shown in figure ($\ref{fig:setup_scheme}$). 
Our plasma chamber consists of a driven cathode, a grounded anode and a grounded wall. 
In between we have our working gas (in our case Oxygen). 
A mass spectrometer located at the anode detects the count rate of incoming ions. 
Despite having symmetric electrodes we have an asymmetrical discharge urging from the grounded wall. 
This asymmetry depends on the gas pressure in the chamber and the applied voltage. 
In other words, the lower the pressure and the higher the voltage, the more the ratio from the grounded area to the driven area rises.   
This affects distribution functions and particle densities close to the electrode edges.
\\
According to the experiment we drive our cathode with a rf voltage $U(t)=U_{rf}\sin{(f_{rf}\cdot t)}$ with $U_{rf}$ ranging from 200$\ein{V}$ to 1000$\ein{V}$ and $f_{rf}=13.56\ein{MHz}$ while our anode stays grounded. \\
In this operational regime the experiment states that the density of negative ions may has an high energetic peak, when measured near the grounded anode. 
That means negative ions $O^-$ have to be produced either in the sheath near the cathode or at the cathode itself. 
The effect apparently gets stronger the lower the pressure becomes due to less detachment collisions (s. table \ref{table:a}). 
We will simulate an equal discharge with particle in cell Monte Carlo collision (PIC-MCC) methods and try to understand which effects cause these 
peaks (s. figure \ref{fig:normal_distribution}).  
This method is described in the next chapter (\ref{blub}).\\
The experiment shows us that the material of the electrode has a huge impact on the count rate of high energetic negative ions. 
With the PIC method it is possible to simulate even complex plasma wall interactions.
Our aim will be to simulate this discharge and eventually see that these high energetic negative ions may as well be produced by the bulk or sheath effects. 
We will, without further knowledge of the exact processes, implement the production of negative ions at the electrode and compare to the experimental results. 
In that case we will have to find a theoretical model to describe the surface process.




Considering the additional reaction channels (in our case with Oxygen as the operating gas) we get our rate equations

\begin{align}
    \frac{\mathrm{d}n_e}{\mathrm{dt}}&=n_e O_2 k_1 + O^- O_2 k_1 - n_e O^- k_2 - n_e O_2^+ k_2 - n_e O_2 k_2 - \nabla \vec{\Phi}_e \nonumber \; ,\\
    \frac{\mathrm{d}O_2^+}{\mathrm{dt}}&=n_e O_2 k_1 - n_e O_2^+ k_2 - O^- O_2^+ k_2 - \nabla \vec{\Phi}_{O_2^+} \nonumber \; , \\
    \frac{\mathrm{d}O^-}{\mathrm{dt}}&=n_e O_2 k_1 - O^- O_2^+ k_2 - O^- O_2 k_2 - \nabla \vec{\Phi}_{O^-} \nonumber \; ,\\
    \vec{\Phi}_i&= q_i\mu_iN_i\vec{E}-D_i\nabla N_i \nonumber \qquad \mathrm{and} \\
    \nabla \vec{E}&= \frac{q}{\epsilon_0}\left( O_2^+ - O^- - n_e\right) \; .
\label{rates}
\end{align}



%taken from basic plasma----------------------------------------------------
The ions have a higher mass than the electrons which results in higher velocities for the electrons than for the ions. 
Due to the high mobility of the electrons plasma shields electric fields. 
Charge separation only exists on scales of the Debye length

\begin{align}
    \lambda_{D,s} = \sqrt{\frac{\varepsilon_0 k_B T_s}{n_s q_s^2}} \; ,
\end{align}

where $\varepsilon_0$ is the dielectric constant in a vacuum, $k_B$ is the Boltzmann constant and $T_s$ is the respective specie temperature. 
The shielding potential of a plasma particle with charge $q$ is given by the Debye-H\"uckel-potential

\begin{align}
    \Phi(r)=\frac{q}{4 \pi \varepsilon_0 r} \mathrm{exp}(-r/\lambda_D) \; ,
\end{align}

Since both electrons and ions contribute to the shielding, the total Debye length is defined by

\begin{align}
    \lambda_D^{-2}= \lambda_{D,e}^{-2} + \lambda_{D,i}^{-2} \; .
\end{align}

but because of the much higher velocity of the electrons, the Debye length can be approximated by $\lambda_D \approx \lambda_{D,e}$.\\ %name of thihs approach ?????
Outside of a sphere with a radius of one Debye-length (Debye sphere) the plasma is considered neutral or "quasi neutral". 
That means that the densities of the different charged particles are close to equal

\begin{align}
    n_0 = n_e =\sum Z_i n_i \; ,
\end{align}

with the ion charge number $Z_i$. The sum considers that there may be more than one ion species, as in electronegative plasmas, which we will discuss later in this thesis (s. chapter \ref{elneg}).\\
The resulting plasma potential $\Phi_{Pl}$ is constant in the bulk region due to the shielding of external fields. \\
Charged particle follow electromagnetic fields. The interaction is called the Lorentz-force $F_L$. 
For a particle with mass $m_s$ and charge $q_s$ in the electrostatic non-relativistic case the equation of motion is given by

\begin{align}
    \dot{\vec{p}}=q_s \vec{E}  \; .
\end{align}

The acceleration of charged particles

\begin{align}
    \vec{a}_s = \frac{q_s}{m_s} \vec{E}
\end{align}

leads to charge separation. This is equal to a small perturbation of the electron density. 
Since the electrons are displaced in respect to the ions the Coulomb force

\begin{align}
    F(r) = \frac{Z_i q_e q_i}{4 \pi \varepsilon_0} \frac{1}{r^2}
\end{align}

pulls them back. Here is $r=|\vec{r}|$ the distance between the electron and the ion. 
Hence the electrons will start to oscillate.
These fast plasma oscillations have an angular frequency of

\begin{align}
    \omega_{p,e} = \sqrt{\frac{n_e e^2}{\varepsilon_0 m_e}} \; ,
\end{align}

the so called electron plasma frequency.

%SIE
We do not claim this to  be correct physically.
But it is a good approximation, because we do not want to compare total numbers with the experiment.
For us it is sufficient to get qualitative results and visual trends.\\
We define the parameter $\eta$ describing the ratio of injected $O^-$ to impinging $O^+_2$.
The anions will be injected with a Maxwell distributed energy. 
We declare that the used method is not exclusive for impinging $O_2^+$ due to the consistent flux $\Gamma_i$ of positive ions towards the anode.
Any other reaction resulting in the injection of negative ions would serve our needs.
Since theoretical studies have not specified the possible reactions yet we stick to our proposed channel of SIE.



%cross sections, maybe put in appendic
\begin{figure}[htbp]
    \centering
    \includegraphics[width=.6\textwidth]{../data/xsct/elastic_xsct.eps}
    \caption{Cross section of oxygen for elastic collisions.}
    \label{fig:coll_elastic}
\end{figure}



\begin{figure}[htbp]
    \centering
    \includegraphics[width=.6\textwidth]{../data/xsct/excite_xsct.eps}
    \caption{Cross section of oxygen for excitation collisions.}
    \label{fig:coll_elastic}
\end{figure}

\begin{figure}[htbp]
    \centering
    \includegraphics[width=.6\textwidth]{../data/xsct/ion_xsct.eps}
    \caption{Cross section of oxygen for ion collisions.}
    \label{fig:coll_neg}
\end{figure}


