\section{Experimental setup}
\label{exp}
Experimental results from the group of Professor Meichsner are used. 
The experimental setup is shown in figure (\ref{fig:setup}). 
It shows a schematic plasma chamber from above. 
The plasma apparatus consists of a cylindrical vacuum chamber built of stainless steel.
At the cathode a rf voltage of frequency 13.56$\ein{Hz}$ and a voltage of (500-2000)$\ein{V}$ is applied.
The anode and the casing are both grounded.
The distance between the electrodes is varied between (3-5)$\ein{cm}$.
The difference of the area of the cathode to the grounded anode in addition to the grounded wall leads to an asymmetric discharge characterized by a negative DC self-bias voltage at the powered electrode. 
The pressure is set up to (2-10)$\ein{Pa}$ and the process gas is oxygen $\rm{O}_2$.
To detect ions $\rm{O}^-$ at the anode a mass spectrometer is attached.
Additionally, a PROES (\textbf{P}hase \textbf{r}esolved \textbf{o}ptical \textbf{e}mission \textbf{s}pectroscopy) diagnostic is applied.\\

\begin{figure}[htbp]
    \begin{minipage}{0.5\textwidth}
       \includegraphics[width=\textwidth]{../pics/Aufbau.eps}
       \caption{Top view of the schematic experimental setup taken from \cite{basti}.}
        \label{fig:setup}
    \end{minipage}
    \begin{minipage}{0.5\textwidth}
            \begin{tabular}{l l}
            \textbf{Operating parameter}: & \\
            frequency : & 13.56$\ein{MHz}$\\
            power : & 7...51$\ein{W}$\\
            pressure : & 4...10$\ein{Pa}$\\
            process gas : & O$_2$\\
                material of electrode : & stainless steel, \\ & SiO$_2$, Al$_2$O$_3$, MgO
            \end{tabular}
    \end{minipage}
\end{figure}

By applying a voltage to the cathode a plasma develops. 
The basics of ccrf plasma physics are described in chapter (\ref{basic_plasma}).
The cold positive ions $\rm{O_2}^{+}$  of the bulk get accelerated at the plasma sheath towards the grounded anode. 
There they are detected by the mass spectrometer and result in a well known distribution \cite{ccrf} as shown in figure (\ref{fig:normal_distribution}). 
\begin{figure}[htbp]
    \centering
    \includegraphics[width=0.8\textwidth]{../pics2/pos_comp.png}
    \caption{Comparison of the positive ion energy distribution for stainless steel and magnesium oxide.}
    \label{fig:normal_distribution}
\end{figure}
The incoming negative ions $\rm{O}^-$ are detected as well. 
The number of detected negative ions is much lower than the number of detected positive ions.
The negative ion distribution of stainless steel and magnesium oxide in figure (\ref{fig:MgO}) show a low-energy peak followed by a plateau which proceeds up to very high energies of several 100$\ein{eV}$. \\
Depending on the cathode material a high-energy peak appears in the negative ion distribution. 
\begin{figure}[htbp]
    \centering
    \includegraphics[width=0.8\textwidth]{../pics2/neg_comp.png}
    \caption{Comparison of the negative ion energy distribution for stainless steel and magnesium oxide.}
    \label{fig:MgO}
\end{figure}

In this experiment different materials have been tested (Si$\rm{O}_2$, Al$_2$O$_3$, magnesium oxide (MgO), stainless steel).
In the following I use a PIC simulation to test if these ions originate from surface processes at the cathode or if they emerge from asymmetry effects.\\



